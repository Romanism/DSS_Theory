
% Default to the notebook output style

    


% Inherit from the specified cell style.




    
\documentclass[11pt]{article}

    
    
    \usepackage[T1]{fontenc}
    % Nicer default font (+ math font) than Computer Modern for most use cases
    \usepackage{mathpazo}

    % Basic figure setup, for now with no caption control since it's done
    % automatically by Pandoc (which extracts ![](path) syntax from Markdown).
    \usepackage{graphicx}
    % We will generate all images so they have a width \maxwidth. This means
    % that they will get their normal width if they fit onto the page, but
    % are scaled down if they would overflow the margins.
    \makeatletter
    \def\maxwidth{\ifdim\Gin@nat@width>\linewidth\linewidth
    \else\Gin@nat@width\fi}
    \makeatother
    \let\Oldincludegraphics\includegraphics
    % Set max figure width to be 80% of text width, for now hardcoded.
    \renewcommand{\includegraphics}[1]{\Oldincludegraphics[width=.8\maxwidth]{#1}}
    % Ensure that by default, figures have no caption (until we provide a
    % proper Figure object with a Caption API and a way to capture that
    % in the conversion process - todo).
    \usepackage{caption}
    \DeclareCaptionLabelFormat{nolabel}{}
    \captionsetup{labelformat=nolabel}

    \usepackage{adjustbox} % Used to constrain images to a maximum size 
    \usepackage{xcolor} % Allow colors to be defined
    \usepackage{enumerate} % Needed for markdown enumerations to work
    \usepackage{geometry} % Used to adjust the document margins
    \usepackage{amsmath} % Equations
    \usepackage{amssymb} % Equations
    \usepackage{textcomp} % defines textquotesingle
    % Hack from http://tex.stackexchange.com/a/47451/13684:
    \AtBeginDocument{%
        \def\PYZsq{\textquotesingle}% Upright quotes in Pygmentized code
    }
    \usepackage{upquote} % Upright quotes for verbatim code
    \usepackage{eurosym} % defines \euro
    \usepackage[mathletters]{ucs} % Extended unicode (utf-8) support
    \usepackage[utf8x]{inputenc} % Allow utf-8 characters in the tex document
    \usepackage{fancyvrb} % verbatim replacement that allows latex
    \usepackage{grffile} % extends the file name processing of package graphics 
                         % to support a larger range 
    % The hyperref package gives us a pdf with properly built
    % internal navigation ('pdf bookmarks' for the table of contents,
    % internal cross-reference links, web links for URLs, etc.)
    \usepackage{hyperref}
    \usepackage{longtable} % longtable support required by pandoc >1.10
    \usepackage{booktabs}  % table support for pandoc > 1.12.2
    \usepackage[inline]{enumitem} % IRkernel/repr support (it uses the enumerate* environment)
    \usepackage[normalem]{ulem} % ulem is needed to support strikethroughs (\sout)
                                % normalem makes italics be italics, not underlines
    

    
    
    % Colors for the hyperref package
    \definecolor{urlcolor}{rgb}{0,.145,.698}
    \definecolor{linkcolor}{rgb}{.71,0.21,0.01}
    \definecolor{citecolor}{rgb}{.12,.54,.11}

    % ANSI colors
    \definecolor{ansi-black}{HTML}{3E424D}
    \definecolor{ansi-black-intense}{HTML}{282C36}
    \definecolor{ansi-red}{HTML}{E75C58}
    \definecolor{ansi-red-intense}{HTML}{B22B31}
    \definecolor{ansi-green}{HTML}{00A250}
    \definecolor{ansi-green-intense}{HTML}{007427}
    \definecolor{ansi-yellow}{HTML}{DDB62B}
    \definecolor{ansi-yellow-intense}{HTML}{B27D12}
    \definecolor{ansi-blue}{HTML}{208FFB}
    \definecolor{ansi-blue-intense}{HTML}{0065CA}
    \definecolor{ansi-magenta}{HTML}{D160C4}
    \definecolor{ansi-magenta-intense}{HTML}{A03196}
    \definecolor{ansi-cyan}{HTML}{60C6C8}
    \definecolor{ansi-cyan-intense}{HTML}{258F8F}
    \definecolor{ansi-white}{HTML}{C5C1B4}
    \definecolor{ansi-white-intense}{HTML}{A1A6B2}

    % commands and environments needed by pandoc snippets
    % extracted from the output of `pandoc -s`
    \providecommand{\tightlist}{%
      \setlength{\itemsep}{0pt}\setlength{\parskip}{0pt}}
    \DefineVerbatimEnvironment{Highlighting}{Verbatim}{commandchars=\\\{\}}
    % Add ',fontsize=\small' for more characters per line
    \newenvironment{Shaded}{}{}
    \newcommand{\KeywordTok}[1]{\textcolor[rgb]{0.00,0.44,0.13}{\textbf{{#1}}}}
    \newcommand{\DataTypeTok}[1]{\textcolor[rgb]{0.56,0.13,0.00}{{#1}}}
    \newcommand{\DecValTok}[1]{\textcolor[rgb]{0.25,0.63,0.44}{{#1}}}
    \newcommand{\BaseNTok}[1]{\textcolor[rgb]{0.25,0.63,0.44}{{#1}}}
    \newcommand{\FloatTok}[1]{\textcolor[rgb]{0.25,0.63,0.44}{{#1}}}
    \newcommand{\CharTok}[1]{\textcolor[rgb]{0.25,0.44,0.63}{{#1}}}
    \newcommand{\StringTok}[1]{\textcolor[rgb]{0.25,0.44,0.63}{{#1}}}
    \newcommand{\CommentTok}[1]{\textcolor[rgb]{0.38,0.63,0.69}{\textit{{#1}}}}
    \newcommand{\OtherTok}[1]{\textcolor[rgb]{0.00,0.44,0.13}{{#1}}}
    \newcommand{\AlertTok}[1]{\textcolor[rgb]{1.00,0.00,0.00}{\textbf{{#1}}}}
    \newcommand{\FunctionTok}[1]{\textcolor[rgb]{0.02,0.16,0.49}{{#1}}}
    \newcommand{\RegionMarkerTok}[1]{{#1}}
    \newcommand{\ErrorTok}[1]{\textcolor[rgb]{1.00,0.00,0.00}{\textbf{{#1}}}}
    \newcommand{\NormalTok}[1]{{#1}}
    
    % Additional commands for more recent versions of Pandoc
    \newcommand{\ConstantTok}[1]{\textcolor[rgb]{0.53,0.00,0.00}{{#1}}}
    \newcommand{\SpecialCharTok}[1]{\textcolor[rgb]{0.25,0.44,0.63}{{#1}}}
    \newcommand{\VerbatimStringTok}[1]{\textcolor[rgb]{0.25,0.44,0.63}{{#1}}}
    \newcommand{\SpecialStringTok}[1]{\textcolor[rgb]{0.73,0.40,0.53}{{#1}}}
    \newcommand{\ImportTok}[1]{{#1}}
    \newcommand{\DocumentationTok}[1]{\textcolor[rgb]{0.73,0.13,0.13}{\textit{{#1}}}}
    \newcommand{\AnnotationTok}[1]{\textcolor[rgb]{0.38,0.63,0.69}{\textbf{\textit{{#1}}}}}
    \newcommand{\CommentVarTok}[1]{\textcolor[rgb]{0.38,0.63,0.69}{\textbf{\textit{{#1}}}}}
    \newcommand{\VariableTok}[1]{\textcolor[rgb]{0.10,0.09,0.49}{{#1}}}
    \newcommand{\ControlFlowTok}[1]{\textcolor[rgb]{0.00,0.44,0.13}{\textbf{{#1}}}}
    \newcommand{\OperatorTok}[1]{\textcolor[rgb]{0.40,0.40,0.40}{{#1}}}
    \newcommand{\BuiltInTok}[1]{{#1}}
    \newcommand{\ExtensionTok}[1]{{#1}}
    \newcommand{\PreprocessorTok}[1]{\textcolor[rgb]{0.74,0.48,0.00}{{#1}}}
    \newcommand{\AttributeTok}[1]{\textcolor[rgb]{0.49,0.56,0.16}{{#1}}}
    \newcommand{\InformationTok}[1]{\textcolor[rgb]{0.38,0.63,0.69}{\textbf{\textit{{#1}}}}}
    \newcommand{\WarningTok}[1]{\textcolor[rgb]{0.38,0.63,0.69}{\textbf{\textit{{#1}}}}}
    
    
    % Define a nice break command that doesn't care if a line doesn't already
    % exist.
    \def\br{\hspace*{\fill} \\* }
    % Math Jax compatability definitions
    \def\gt{>}
    \def\lt{<}
    % Document parameters
    \title{10\_Variance\_standard\_deviation\_moment}
    
    
    

    % Pygments definitions
    
\makeatletter
\def\PY@reset{\let\PY@it=\relax \let\PY@bf=\relax%
    \let\PY@ul=\relax \let\PY@tc=\relax%
    \let\PY@bc=\relax \let\PY@ff=\relax}
\def\PY@tok#1{\csname PY@tok@#1\endcsname}
\def\PY@toks#1+{\ifx\relax#1\empty\else%
    \PY@tok{#1}\expandafter\PY@toks\fi}
\def\PY@do#1{\PY@bc{\PY@tc{\PY@ul{%
    \PY@it{\PY@bf{\PY@ff{#1}}}}}}}
\def\PY#1#2{\PY@reset\PY@toks#1+\relax+\PY@do{#2}}

\expandafter\def\csname PY@tok@w\endcsname{\def\PY@tc##1{\textcolor[rgb]{0.73,0.73,0.73}{##1}}}
\expandafter\def\csname PY@tok@c\endcsname{\let\PY@it=\textit\def\PY@tc##1{\textcolor[rgb]{0.25,0.50,0.50}{##1}}}
\expandafter\def\csname PY@tok@cp\endcsname{\def\PY@tc##1{\textcolor[rgb]{0.74,0.48,0.00}{##1}}}
\expandafter\def\csname PY@tok@k\endcsname{\let\PY@bf=\textbf\def\PY@tc##1{\textcolor[rgb]{0.00,0.50,0.00}{##1}}}
\expandafter\def\csname PY@tok@kp\endcsname{\def\PY@tc##1{\textcolor[rgb]{0.00,0.50,0.00}{##1}}}
\expandafter\def\csname PY@tok@kt\endcsname{\def\PY@tc##1{\textcolor[rgb]{0.69,0.00,0.25}{##1}}}
\expandafter\def\csname PY@tok@o\endcsname{\def\PY@tc##1{\textcolor[rgb]{0.40,0.40,0.40}{##1}}}
\expandafter\def\csname PY@tok@ow\endcsname{\let\PY@bf=\textbf\def\PY@tc##1{\textcolor[rgb]{0.67,0.13,1.00}{##1}}}
\expandafter\def\csname PY@tok@nb\endcsname{\def\PY@tc##1{\textcolor[rgb]{0.00,0.50,0.00}{##1}}}
\expandafter\def\csname PY@tok@nf\endcsname{\def\PY@tc##1{\textcolor[rgb]{0.00,0.00,1.00}{##1}}}
\expandafter\def\csname PY@tok@nc\endcsname{\let\PY@bf=\textbf\def\PY@tc##1{\textcolor[rgb]{0.00,0.00,1.00}{##1}}}
\expandafter\def\csname PY@tok@nn\endcsname{\let\PY@bf=\textbf\def\PY@tc##1{\textcolor[rgb]{0.00,0.00,1.00}{##1}}}
\expandafter\def\csname PY@tok@ne\endcsname{\let\PY@bf=\textbf\def\PY@tc##1{\textcolor[rgb]{0.82,0.25,0.23}{##1}}}
\expandafter\def\csname PY@tok@nv\endcsname{\def\PY@tc##1{\textcolor[rgb]{0.10,0.09,0.49}{##1}}}
\expandafter\def\csname PY@tok@no\endcsname{\def\PY@tc##1{\textcolor[rgb]{0.53,0.00,0.00}{##1}}}
\expandafter\def\csname PY@tok@nl\endcsname{\def\PY@tc##1{\textcolor[rgb]{0.63,0.63,0.00}{##1}}}
\expandafter\def\csname PY@tok@ni\endcsname{\let\PY@bf=\textbf\def\PY@tc##1{\textcolor[rgb]{0.60,0.60,0.60}{##1}}}
\expandafter\def\csname PY@tok@na\endcsname{\def\PY@tc##1{\textcolor[rgb]{0.49,0.56,0.16}{##1}}}
\expandafter\def\csname PY@tok@nt\endcsname{\let\PY@bf=\textbf\def\PY@tc##1{\textcolor[rgb]{0.00,0.50,0.00}{##1}}}
\expandafter\def\csname PY@tok@nd\endcsname{\def\PY@tc##1{\textcolor[rgb]{0.67,0.13,1.00}{##1}}}
\expandafter\def\csname PY@tok@s\endcsname{\def\PY@tc##1{\textcolor[rgb]{0.73,0.13,0.13}{##1}}}
\expandafter\def\csname PY@tok@sd\endcsname{\let\PY@it=\textit\def\PY@tc##1{\textcolor[rgb]{0.73,0.13,0.13}{##1}}}
\expandafter\def\csname PY@tok@si\endcsname{\let\PY@bf=\textbf\def\PY@tc##1{\textcolor[rgb]{0.73,0.40,0.53}{##1}}}
\expandafter\def\csname PY@tok@se\endcsname{\let\PY@bf=\textbf\def\PY@tc##1{\textcolor[rgb]{0.73,0.40,0.13}{##1}}}
\expandafter\def\csname PY@tok@sr\endcsname{\def\PY@tc##1{\textcolor[rgb]{0.73,0.40,0.53}{##1}}}
\expandafter\def\csname PY@tok@ss\endcsname{\def\PY@tc##1{\textcolor[rgb]{0.10,0.09,0.49}{##1}}}
\expandafter\def\csname PY@tok@sx\endcsname{\def\PY@tc##1{\textcolor[rgb]{0.00,0.50,0.00}{##1}}}
\expandafter\def\csname PY@tok@m\endcsname{\def\PY@tc##1{\textcolor[rgb]{0.40,0.40,0.40}{##1}}}
\expandafter\def\csname PY@tok@gh\endcsname{\let\PY@bf=\textbf\def\PY@tc##1{\textcolor[rgb]{0.00,0.00,0.50}{##1}}}
\expandafter\def\csname PY@tok@gu\endcsname{\let\PY@bf=\textbf\def\PY@tc##1{\textcolor[rgb]{0.50,0.00,0.50}{##1}}}
\expandafter\def\csname PY@tok@gd\endcsname{\def\PY@tc##1{\textcolor[rgb]{0.63,0.00,0.00}{##1}}}
\expandafter\def\csname PY@tok@gi\endcsname{\def\PY@tc##1{\textcolor[rgb]{0.00,0.63,0.00}{##1}}}
\expandafter\def\csname PY@tok@gr\endcsname{\def\PY@tc##1{\textcolor[rgb]{1.00,0.00,0.00}{##1}}}
\expandafter\def\csname PY@tok@ge\endcsname{\let\PY@it=\textit}
\expandafter\def\csname PY@tok@gs\endcsname{\let\PY@bf=\textbf}
\expandafter\def\csname PY@tok@gp\endcsname{\let\PY@bf=\textbf\def\PY@tc##1{\textcolor[rgb]{0.00,0.00,0.50}{##1}}}
\expandafter\def\csname PY@tok@go\endcsname{\def\PY@tc##1{\textcolor[rgb]{0.53,0.53,0.53}{##1}}}
\expandafter\def\csname PY@tok@gt\endcsname{\def\PY@tc##1{\textcolor[rgb]{0.00,0.27,0.87}{##1}}}
\expandafter\def\csname PY@tok@err\endcsname{\def\PY@bc##1{\setlength{\fboxsep}{0pt}\fcolorbox[rgb]{1.00,0.00,0.00}{1,1,1}{\strut ##1}}}
\expandafter\def\csname PY@tok@kc\endcsname{\let\PY@bf=\textbf\def\PY@tc##1{\textcolor[rgb]{0.00,0.50,0.00}{##1}}}
\expandafter\def\csname PY@tok@kd\endcsname{\let\PY@bf=\textbf\def\PY@tc##1{\textcolor[rgb]{0.00,0.50,0.00}{##1}}}
\expandafter\def\csname PY@tok@kn\endcsname{\let\PY@bf=\textbf\def\PY@tc##1{\textcolor[rgb]{0.00,0.50,0.00}{##1}}}
\expandafter\def\csname PY@tok@kr\endcsname{\let\PY@bf=\textbf\def\PY@tc##1{\textcolor[rgb]{0.00,0.50,0.00}{##1}}}
\expandafter\def\csname PY@tok@bp\endcsname{\def\PY@tc##1{\textcolor[rgb]{0.00,0.50,0.00}{##1}}}
\expandafter\def\csname PY@tok@fm\endcsname{\def\PY@tc##1{\textcolor[rgb]{0.00,0.00,1.00}{##1}}}
\expandafter\def\csname PY@tok@vc\endcsname{\def\PY@tc##1{\textcolor[rgb]{0.10,0.09,0.49}{##1}}}
\expandafter\def\csname PY@tok@vg\endcsname{\def\PY@tc##1{\textcolor[rgb]{0.10,0.09,0.49}{##1}}}
\expandafter\def\csname PY@tok@vi\endcsname{\def\PY@tc##1{\textcolor[rgb]{0.10,0.09,0.49}{##1}}}
\expandafter\def\csname PY@tok@vm\endcsname{\def\PY@tc##1{\textcolor[rgb]{0.10,0.09,0.49}{##1}}}
\expandafter\def\csname PY@tok@sa\endcsname{\def\PY@tc##1{\textcolor[rgb]{0.73,0.13,0.13}{##1}}}
\expandafter\def\csname PY@tok@sb\endcsname{\def\PY@tc##1{\textcolor[rgb]{0.73,0.13,0.13}{##1}}}
\expandafter\def\csname PY@tok@sc\endcsname{\def\PY@tc##1{\textcolor[rgb]{0.73,0.13,0.13}{##1}}}
\expandafter\def\csname PY@tok@dl\endcsname{\def\PY@tc##1{\textcolor[rgb]{0.73,0.13,0.13}{##1}}}
\expandafter\def\csname PY@tok@s2\endcsname{\def\PY@tc##1{\textcolor[rgb]{0.73,0.13,0.13}{##1}}}
\expandafter\def\csname PY@tok@sh\endcsname{\def\PY@tc##1{\textcolor[rgb]{0.73,0.13,0.13}{##1}}}
\expandafter\def\csname PY@tok@s1\endcsname{\def\PY@tc##1{\textcolor[rgb]{0.73,0.13,0.13}{##1}}}
\expandafter\def\csname PY@tok@mb\endcsname{\def\PY@tc##1{\textcolor[rgb]{0.40,0.40,0.40}{##1}}}
\expandafter\def\csname PY@tok@mf\endcsname{\def\PY@tc##1{\textcolor[rgb]{0.40,0.40,0.40}{##1}}}
\expandafter\def\csname PY@tok@mh\endcsname{\def\PY@tc##1{\textcolor[rgb]{0.40,0.40,0.40}{##1}}}
\expandafter\def\csname PY@tok@mi\endcsname{\def\PY@tc##1{\textcolor[rgb]{0.40,0.40,0.40}{##1}}}
\expandafter\def\csname PY@tok@il\endcsname{\def\PY@tc##1{\textcolor[rgb]{0.40,0.40,0.40}{##1}}}
\expandafter\def\csname PY@tok@mo\endcsname{\def\PY@tc##1{\textcolor[rgb]{0.40,0.40,0.40}{##1}}}
\expandafter\def\csname PY@tok@ch\endcsname{\let\PY@it=\textit\def\PY@tc##1{\textcolor[rgb]{0.25,0.50,0.50}{##1}}}
\expandafter\def\csname PY@tok@cm\endcsname{\let\PY@it=\textit\def\PY@tc##1{\textcolor[rgb]{0.25,0.50,0.50}{##1}}}
\expandafter\def\csname PY@tok@cpf\endcsname{\let\PY@it=\textit\def\PY@tc##1{\textcolor[rgb]{0.25,0.50,0.50}{##1}}}
\expandafter\def\csname PY@tok@c1\endcsname{\let\PY@it=\textit\def\PY@tc##1{\textcolor[rgb]{0.25,0.50,0.50}{##1}}}
\expandafter\def\csname PY@tok@cs\endcsname{\let\PY@it=\textit\def\PY@tc##1{\textcolor[rgb]{0.25,0.50,0.50}{##1}}}

\def\PYZbs{\char`\\}
\def\PYZus{\char`\_}
\def\PYZob{\char`\{}
\def\PYZcb{\char`\}}
\def\PYZca{\char`\^}
\def\PYZam{\char`\&}
\def\PYZlt{\char`\<}
\def\PYZgt{\char`\>}
\def\PYZsh{\char`\#}
\def\PYZpc{\char`\%}
\def\PYZdl{\char`\$}
\def\PYZhy{\char`\-}
\def\PYZsq{\char`\'}
\def\PYZdq{\char`\"}
\def\PYZti{\char`\~}
% for compatibility with earlier versions
\def\PYZat{@}
\def\PYZlb{[}
\def\PYZrb{]}
\makeatother


    % Exact colors from NB
    \definecolor{incolor}{rgb}{0.0, 0.0, 0.5}
    \definecolor{outcolor}{rgb}{0.545, 0.0, 0.0}



    
    % Prevent overflowing lines due to hard-to-break entities
    \sloppy 
    % Setup hyperref package
    \hypersetup{
      breaklinks=true,  % so long urls are correctly broken across lines
      colorlinks=true,
      urlcolor=urlcolor,
      linkcolor=linkcolor,
      citecolor=citecolor,
      }
    % Slightly bigger margins than the latex defaults
    
    \geometry{verbose,tmargin=1in,bmargin=1in,lmargin=1in,rmargin=1in}
    
    

    \begin{document}
    
    
    \maketitle
    
    

    
    \hypertarget{uxbd84uxc0b0uxacfc-uxd45cuxc900uxd3b8uxcc28}{%
\subsubsection{분산과
표준편차}\label{uxbd84uxc0b0uxacfc-uxd45cuxc900uxd3b8uxcc28}}

    분산은 분포의 폭(width)을 대표하는 값 (산포도 - 퍼져있는 정도)

    \[ \text{표준 편차} = \sqrt{\text{분산}} \]

    \hypertarget{uxc0d8uxd50c-uxbd84uxc0b0-uxd45cuxbcf8-uxbd84uxc0b0}{%
\subsubsection{1. 샘플 분산 (표본
분산)}\label{uxc0d8uxd50c-uxbd84uxc0b0-uxd45cuxbcf8-uxbd84uxc0b0}}

    \begin{enumerate}
\def\labelenumi{\arabic{enumi})}
\tightlist
\item
  편향 샘플 분산(biased sample variance)
\end{enumerate}

    \[ s^2 = \dfrac{1}{N}\sum_{i=1}^{N} (x_i-m)^2\;\;\rightarrow (편차)^2의\;평균 \]

    \begin{itemize}
\tightlist
\item
  편향 - 원래 있어야 할 곳에 있지 않고 치우쳤을 때
\item
  편차에 제곱을 하는 이유? - 편차를 다 더하면 0이 되기 때문에 의미 있는
  숫자를 만들지 못함
\end{itemize}

    \begin{enumerate}
\def\labelenumi{\arabic{enumi})}
\setcounter{enumi}{1}
\tightlist
\item
  비 편향 샘플 분산(Unbiased sample variance)
\end{enumerate}

    \[ s^2_{\text{unbiased}} = \dfrac{1}{N-1}\sum_{i=1}^{N} (x_i-m)^2 \]

    \begin{itemize}
\tightlist
\item
  왜 N-1일까? 모 분산이 샘플 분산보다 분산이 크기 때문에
\item
  모분산과 값을 최대한 맞추기 위해선 값을 높여야 함
\end{itemize}

    \hypertarget{uxd655uxb960-uxbd84uxd3ecuxc758-uxbd84uxc0b0-uxbaa8-uxbd84uxc0b0}{%
\subsubsection{2. 확률 분포의 분산 (모
분산)}\label{uxd655uxb960-uxbd84uxd3ecuxc758-uxbd84uxc0b0-uxbaa8-uxbd84uxc0b0}}

    \begin{enumerate}
\def\labelenumi{\arabic{enumi})}
\tightlist
\item
  정의
\end{enumerate}

    {[}1{]} 연속 확률 변수

\[ \sigma^2 = \text{Var}[X] = \text{E}[(X - \mu)^2] = \int_{-\infty}^{\infty} (x - \mu)^2 f(x)dx\]

    {[}2{]} 이산 확률 변수

\[ \sigma^2 = \text{Var}[X] = \text{E}[(X - \mu)^2] =  \sum (x - \mu)^2 P(x) \]

    \hypertarget{uxbd84uxc0b0uxc758-uxc131uxc9c8}{%
\subsubsection{3. 분산의 성질}\label{uxbd84uxc0b0uxc758-uxc131uxc9c8}}

    \begin{enumerate}
\def\labelenumi{\arabic{enumi})}
\tightlist
\item
  \$ \text{Var}{[}X{]} \geq 0 \$
\end{enumerate}

    \begin{enumerate}
\def\labelenumi{\arabic{enumi})}
\setcounter{enumi}{1}
\tightlist
\item
  \$ \text{Var}{[}c{]} = 0 \$
\end{enumerate}

    \begin{enumerate}
\def\labelenumi{\arabic{enumi})}
\setcounter{enumi}{2}
\tightlist
\item
  \$ \text{Var}{[}cX{]} = c\^{}2 \text{Var}{[}X{]} \$
\end{enumerate}

    \begin{enumerate}
\def\labelenumi{\arabic{enumi})}
\setcounter{enumi}{3}
\tightlist
\item
  \$ \text{Var}{[}X{]} = \text{E}{[}X\^{}2{]} - (\text{E}{[}X{]})\^{}2 =
  \text{E}{[}X\^{}2{]} - \mu\^{}2\$ (제평마평제)
\end{enumerate}

    \begin{enumerate}
\def\labelenumi{\arabic{enumi})}
\setcounter{enumi}{4}
\tightlist
\item
  \$ \text{E}{[}X\^{}2{]} = \mu\^{}2 + \text{Var}{[}X{]} \$
\end{enumerate}

    (증명)

\[ 
\begin{eqnarray}
\text{Var}[X] 
&=& \text{E}[(X - \mu)^2] \\\\
&=& \text{E}[X^2 - 2\mu X + \mu^2] \\\\
&=& \text{E}[X^2] - 2\mu\text{E}[X] + \mu^2 \\\\
&=& \text{E}[X^2] - 2\mu^2 + \mu^2 \\\\
&=& \text{E}[X^2] - \mu^2\\\\
\end{eqnarray}
\]

    \hypertarget{uxb450-uxd655uxb960-uxbcc0uxc218uxc758-uxd569uxc758-uxbd84uxc0b0}{%
\subsubsection{4. 두 확률 변수의 합의
분산}\label{uxb450-uxd655uxb960-uxbcc0uxc218uxc758-uxd569uxc758-uxbd84uxc0b0}}

    \begin{enumerate}
\def\labelenumi{\arabic{enumi})}
\tightlist
\item
  독립이 아닌경우
\end{enumerate}

    \[\text{Var}\left[ X + Y \right] = \text{E}\left[ (X -\mu_X)^2 \right] + \text{E}\left[ (Y - \mu_Y)^2 \right] + 2\text{E}\left[ (X-\mu_X)(Y-\mu_Y) \right]\]

    (증명)

\[\begin{eqnarray}
\text{Var}\left[ X + Y \right] 
&=& \text{E}\left[ (X + Y - (\mu_X + \mu_Y))^2 \right] \\\\
&=& \text{E}\left[ ((X -\mu_X) + (Y - \mu_Y))^2 \right] \\\\
&=& \text{E}\left[ (X -\mu_X)^2 + (Y - \mu_Y)^2 + 2(X-\mu_X)(Y-\mu_Y) \right] \\\\
&=& \text{E}\left[ (X -\mu_X)^2 \right] + \text{E}\left[ (Y - \mu_Y)^2 \right] + 2\text{E}\left[ (X-\mu_X)(Y-\mu_Y) \right] 
\end{eqnarray}\]

    \begin{enumerate}
\def\labelenumi{\arabic{enumi})}
\setcounter{enumi}{1}
\tightlist
\item
  독립인 경우
\end{enumerate}

    \[\text{Var}\left[ X + Y \right] = \text{Var}\left[ X \right] + \text{Var}\left[ Y \right] = \text{E}\left[ (X -\mu_X)^2 \right] + \text{E}\left[ (Y - \mu_Y)^2 \right]\]

    \[ \because \text{E}\left[ (X-\mu_X)(Y-\mu_Y) \right] = 0\]

    \hypertarget{uxc0d8uxd50c-uxd3c9uxade0uxc758-uxbd84uxc0b0}{%
\subsubsection{5. 샘플 평균의
분산}\label{uxc0d8uxd50c-uxd3c9uxade0uxc758-uxbd84uxc0b0}}

    \begin{enumerate}
\def\labelenumi{\arabic{enumi})}
\tightlist
\item
  공식 : 샘플 평균의 분산은 모 평균 분산에 (N-1)을 나눠줌
\end{enumerate}

    \[ 
\begin{eqnarray}
\text{Var}[\bar{X}] = \dfrac{\sigma^2}{N-1}\end{eqnarray}
\]

    \begin{enumerate}
\def\labelenumi{\arabic{enumi})}
\setcounter{enumi}{1}
\tightlist
\item
  증명
\end{enumerate}

    \[ 
\begin{eqnarray}
\text{Var}[\bar{X}] 
&=& \text{Var} \left[ \dfrac{1}{N} \sum_{i=1}^N X_i \right] \\\\
&=& \text{E} \left[ \left( \dfrac{1}{N} \sum_{i=1}^N X_i - \mu \right)^2 \right] \\\\
&=& \text{E} \left[ \left( \dfrac{1}{N} \sum_{i=1}^N X_i - \dfrac{1}{N}N\mu \right)^2 \right] \\\\
&=& \text{E} \left[ \left( \dfrac{1}{N} \left( \sum_{i=1}^N X_i - N\mu \right) \right)^2 \right] \\\\
&=& \text{E} \left[ \left( \dfrac{1}{N} \sum_{i=1}^N (X_i - \mu) \right)^2 \right] \\\\
&=& \text{E} \left[ \dfrac{1}{N^2} \sum_{i=1}^N \sum_{j=1}^N (X_i - \mu) (X_j - \mu)  \right] \\\\\end{eqnarray}\]

    \$X\_i, X\_j가 독립일 때,
\$\(\text{E}\left[ (X_i-\mu)(X_j-\mu) \right] = 0\)를 활용하면 \(i=j\)인
항, 즉 제곱항만 남음

    \[ 
\begin{eqnarray}
&=& \text{E} \left[ \dfrac{1}{N^2} \sum_{i=1}^N (X_i - \mu)^2 \right] \\\\
&=& \text{E} \left[ \dfrac{1}{N^2} \sum_{i=1}^N (X - \mu)^2 \right] \\\\
&=& \text{E} \left[ \dfrac{1}{N^2} N (X - \mu)^2 \right] \\\\
&=& \text{E} \left[ \dfrac{1}{N} (X - \mu)^2 \right] \\\\
&=& \dfrac{1}{N} \text{Var}[X] = \dfrac{\sigma^2}{N}
\end{eqnarray}
\]

    \hypertarget{uxc0d8uxd50c-uxbd84uxc0b0uxc758-uxae30uxb300uxac12}{%
\subsubsection{6. 샘플 분산의
기대값}\label{uxc0d8uxd50c-uxbd84uxc0b0uxc758-uxae30uxb300uxac12}}

    \begin{enumerate}
\def\labelenumi{\arabic{enumi})}
\tightlist
\item
  샘플 분산의 기대값은 모분산의 \$ \dfrac{N-1}{N}\$
\end{enumerate}

    \[\text{E}[s^2] = \dfrac{N-1}{N}\sigma^2\]

    \begin{enumerate}
\def\labelenumi{\arabic{enumi})}
\setcounter{enumi}{1}
\tightlist
\item
  증명
\end{enumerate}

    \[ \begin{eqnarray}
\text{E}[s^2] 
&=& \text{E} \left[ \dfrac{1}{N}\sum_{i=1}^N (X_i - \bar{X})^2 \right] 
= \text{E} \left[ \dfrac{1}{N}\sum_{i=1}^N \left\{ (X_i -\mu) - (\bar{X} - \mu) \right \}^2  \right] \\\\
&=& \text{E} \left[ \dfrac{1}{N}\sum_{i=1}^N \left\{ (X_i -\mu)^2 - 2 (X_i -\mu)(\bar{X} - \mu) +  (\bar{X} - \mu)^2 \right \}  \right] \\\\
&=&  \text{E} \left[ \dfrac{1}{N} \sum_{i=1}^N (X_i -\mu)^2 \right] 
  - 2 \text{E} \left[ { \dfrac{1}{N} \sum_{i=1}^N (X_i -\mu)(\bar{X} - \mu) } \right] 
  +  \text{E} \left[ { \dfrac{1}{N} \sum_{i=1}^N (\bar{X} - \mu)^2 } \right] 
\end{eqnarray} \]

    {[}1{]} 첫번째 항

    \[ \begin{eqnarray}
\text{E}[s^2] 
&=& \text{E} \left[ \dfrac{1}{N}\sum_{i=1}^N (X_i - \bar{X})^2 \right] 
= \text{E} \left[ \dfrac{1}{N}\sum_{i=1}^N \left\{ (X_i -\mu) - (\bar{X} - \mu) \right \}^2  \right] \\\\
&=& \text{E} \left[ \dfrac{1}{N}\sum_{i=1}^N \left\{ (X_i -\mu)^2 - 2 (X_i -\mu)(\bar{X} - \mu) +  (\bar{X} - \mu)^2 \right \}  \right] \\\\
&=&  \text{E} \left[ \dfrac{1}{N} \sum_{i=1}^N (X_i -\mu)^2 \right] 
  - 2 \text{E} \left[ { \dfrac{1}{N} \sum_{i=1}^N (X_i -\mu)(\bar{X} - \mu) } \right] 
  +  \text{E} \left[ { \dfrac{1}{N} \sum_{i=1}^N (\bar{X} - \mu)^2 } \right] 
\end{eqnarray} \]

    {[}2{]} 두번째 항

    \[ \begin{eqnarray}
\text{E} \left[ { \dfrac{1}{N} \sum_{i=1}^N (X_i -\mu)(\bar{X} - \mu) } \right] 
&=& \text{E} \left[ { \dfrac{1}{N} \sum_{i=1}^N (X_i -\mu) \left( \dfrac{1}{N} \sum_{j=1}^N X_j - \mu \right) } \right]  \\\\
&=& \text{E} \left[ { \dfrac{1}{N} \sum_{i=1}^N (X_i -\mu) \left( \dfrac{1}{N} \sum_{j=1}^N ( X_j - \mu ) \right) } \right]  \\\\
&=& \text{E} \left[ { \dfrac{1}{N^2} \sum_{i=1}^N \sum_{j=1}^N (X_i -\mu)  ( X_j - \mu )} \right]  \\\\
&=& \dfrac{1}{N} \text{Var}[X] \\\\
&=& \dfrac{\sigma^2}{N}
\end{eqnarray} \]

    {[}3{]} 세번째 항

    \[\text{E} \left[ { \dfrac{1}{N} \sum_{i=1}^N (\bar{X} - \mu)^2 } \right] = \dfrac{1}{N} \text{Var}[X] = \dfrac{\sigma^2}{N}\]

    {[}4{]} 종합

\[
\sigma^2 
= \dfrac{N}{N-1} \text{E}[s^2] 
= \dfrac{N}{N-1} \text{E} \left[ \dfrac{1}{N} \sum (X_i-\bar{X})^2 \right] 
= \text{E} \left[ \dfrac{1}{N-1} \sum (X_i-\bar{X})^2 \right] 
= \text{E} \left[ s^2_{\text{unbiased}} \right]
\]

    \begin{Verbatim}[commandchars=\\\{\}]
{\color{incolor}In [{\color{incolor}6}]:} \PY{c+c1}{\PYZsh{} 분산 구하기}
        \PY{k+kn}{import} \PY{n+nn}{scipy} \PY{k}{as} \PY{n+nn}{sp}
        \PY{k+kn}{import} \PY{n+nn}{scipy}\PY{n+nn}{.}\PY{n+nn}{stats}
        \PY{n}{sp}\PY{o}{.}\PY{n}{random}\PY{o}{.}\PY{n}{seed}\PY{p}{(}\PY{l+m+mi}{0}\PY{p}{)}
        \PY{n}{x} \PY{o}{=} \PY{n}{sp}\PY{o}{.}\PY{n}{stats}\PY{o}{.}\PY{n}{norm}\PY{p}{(}\PY{l+m+mi}{0}\PY{p}{,} \PY{l+m+mi}{2}\PY{p}{)}\PY{o}{.}\PY{n}{rvs}\PY{p}{(}\PY{l+m+mi}{1000}\PY{p}{)}  \PY{c+c1}{\PYZsh{} mean=0, standard deviation=2}
\end{Verbatim}


    \begin{Verbatim}[commandchars=\\\{\}]
{\color{incolor}In [{\color{incolor}11}]:} \PY{n}{np}\PY{o}{.}\PY{n}{var}\PY{p}{(}\PY{n}{x}\PY{p}{)}
\end{Verbatim}


\begin{Verbatim}[commandchars=\\\{\}]
{\color{outcolor}Out[{\color{outcolor}11}]:} 3.896937825248617
\end{Verbatim}
            
    \begin{Verbatim}[commandchars=\\\{\}]
{\color{incolor}In [{\color{incolor}8}]:} \PY{n}{np}\PY{o}{.}\PY{n}{var}\PY{p}{(}\PY{n}{x}\PY{p}{,} \PY{n}{ddof}\PY{o}{=}\PY{l+m+mi}{1}\PY{p}{)}  \PY{c+c1}{\PYZsh{} ddof \PYZhy{} degree of fredom, unbiased variance}
\end{Verbatim}


\begin{Verbatim}[commandchars=\\\{\}]
{\color{outcolor}Out[{\color{outcolor}8}]:} 3.900838663912529
\end{Verbatim}
            
    Q. 왜 표본 분산은 모 분산보다 작을까?

    모 분산의 기준점을 확실히 모르기 때문임.. (자유도 만큼 빼준다)

    \hypertarget{uxbaa8uxba58uxd2b8-moment}{%
\subsubsection{7. 모멘트 (Moment)}\label{uxbaa8uxba58uxd2b8-moment}}

    \begin{enumerate}
\def\labelenumi{\arabic{enumi})}
\tightlist
\item
  정의
\end{enumerate}

    \begin{itemize}
\tightlist
\item
  앞서 구한 기댓값, 분산 등의 특징값을 확률분포의 모멘트라 함
\item
  확률 분포에서 찾아낼 수 있는 특징값의 집합
\end{itemize}

    \begin{enumerate}
\def\labelenumi{\arabic{enumi})}
\setcounter{enumi}{1}
\tightlist
\item
  종류
\end{enumerate}

    {[}1{]} 1차 모멘트 = \(\text{E}[X]\) : 기댓값 (Expectation)

{[}2{]} 2차 모멘트 = \(\text{E}[(X-\mu)^2]\) : 분산 (Variance)

{[}3{]} 3차 모멘트 = \(\text{E}[(X-\mu)^3]\) : 스큐니스 (Skewness), 왜도

{[}4{]} 4차 모멘트 = \(\text{E}[(X-\mu)^4]\) : 커토시스 (Kurtosis), 첨도

    \begin{enumerate}
\def\labelenumi{\arabic{enumi})}
\setcounter{enumi}{2}
\tightlist
\item
  3차 모멘트의 특징
\end{enumerate}

    {[}1{]} 스큐니스가 0이면 대칭인 확률 분포

{[}2{]} 양수이면 오른쪽으로 치우쳐진 확률 분포

    \begin{enumerate}
\def\labelenumi{\arabic{enumi})}
\setcounter{enumi}{2}
\tightlist
\item
  모멘트와 확률 분포
\end{enumerate}

    \[
\begin{eqnarray}
\text{E}[X] &=& \text{E}[Y] \\
\text{E}[(X-\mu_X)^2] &=& \text{E}[(Y-\mu_Y)^2] \\
\text{E}[(X-\mu_X)^3] &=& \text{E}[(Y-\mu_Y)^3] \\
\text{E}[(X-\mu_X)^4] &=& \text{E}[(Y-\mu_Y)^4] \\
\text{E}[(X-\mu_X)^5] &=& \text{E}[(Y-\mu_Y)^5] \\
\vdots &=& \vdots \\
\end{eqnarray}
\]

이면

\[ X = Y \]

    {[}1{]} 두 개의 확률 분포가 있고

{[}2{]} 1차부터 무한대 차수에 이르기까지 두 확률 분포의 모든 모멘트 값이
같으면

{[}3{]} 두 확률 분포는 같은 확률 분포


    % Add a bibliography block to the postdoc
    
    
    
    \end{document}
